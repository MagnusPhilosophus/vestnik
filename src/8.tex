\clearpage
{\bfseries IRSTI 65.53.03}

\section{SAFETY STUDY OF UZBEKISTAN FRESHWATER FISH AND THEIR CANNED FISH}

\begin{center}
{\bfseries I.B. Pulatov\textsuperscript{1*}, K.M
Zhuraeva\textsuperscript{2}, K.O Dodaev\textsuperscript{2}, Kh.N.
Niyozov\textsuperscript{2}}

\textsuperscript{1}University of Veterinary Medicine, Animal Husbandry
and Biotechnology,

Samarkand, Uzbekistan, {\bfseries \textsuperscript{2}}Institute of Chemical
Technology, Tashkent, Uzbekistan,

\href{mailto:pulatov.i1990@mail.ru}{\nolinkurl{pulatov.i1990@mail.ru}}
\end{center}

Methods for industrial processing of freshwater fish, ensuring the
safety of canned fish, quantitative and qualitative determination of the
chemical composition of the tested range of fish and canned food from
them, including proteins with an essential amino acid composition,
carbohydrates, fats, minerals, vitamins, enzymes, detection of hazardous
components, including heavy metals and their salts, conversion products
of pesticides, herbicides, antibiotics, the presence of radionuclides
using gas-liquid chromatography methods, ways to eliminate them, finding
ways to eliminate them, thereby ensuring the food safety requirements of
a particular fish variety, or identifying a hazard, hidden in one or
another canned fish, development of suitability criteria for canning
fish of a particular variety, depending on the area where the fish is
grown, the composition of groundwater and air of a given lake, river or
artificial reservoir, as well as the method of preservation and the
composition of auxiliary materials, such as sauce tomato, vegetable oil,
used in canning fish, which is the scientific novelty and practical
value of this research work, which ultimately allows you to create a map
of the use of freshwater fish production in Uzbekistan, canning methods,
compiling a list of ingredients for canned food for the production of
canned fish, development of individual technology for production.

{\bfseries Keywords:} presswater fish, roach, silver carp, carp, catfish,
safety, proteins, fats, carbohydrates, vitamins, heavy metals.

\begin{center}
{\large\bfseries ӨЗБЕКСТАН ТҰШЫ СУ БАЛЫҚТАРЫНЫҢ ЖӘНЕ ОНЫҢ КОНСЕРВЕРЛЕРІНІҢ
ҚАУІПСІЗДІГІН ЗЕРТТЕУ}

{\bfseries И.Б. Пулатов\textsuperscript{1*}, Қ.М.
Жураева\textsuperscript{2}, Қ.О. Додаев\textsuperscript{2},
Х.Н.Ниёзов\textsuperscript{2}}

\textsuperscript{1}Ветеринария, мал шаруашылығы және биотехнология
университеті, Өзбекстан,

Самарқанд, \textsuperscript{2}Химиялық технология институты, Ташкент,
Өзбекстан,

\href{mailto:pulatov.i1990@mail.ru}{\nolinkurl{pulatov.i1990@mail.ru}},
\end{center}

Тұщы су балығын өнеркәсіптік өңдеу әдістері, балық консервілерінің
қауіпсіздігін қамтамасыз ету, балық пен консервілердің сыналатын
ассортиментінің химиялық құрамын, оның ішінде алмастырылмайтын
аминқышқылдық құрамы бар ақуыздарды, көмірсуларды, майларды, минералды
заттарды сандық және сапалық анықтау, дәрумендер, ферменттер, қауіпті
компоненттерді, оның ішінде ауыр металдар мен олардың тұздарын,
пестицидтердің, гербицидтердің, антибиотиктердің конверсиялық өнімдерін
анықтау, газ-сұйықтық хроматография әдістерін қолдану арқылы
радионуклидтердің болуы, оларды жою жолдары, оларды жою жолдарын табу,
сол арқылы белгілі бір балық сортының азық-түлік қауіпсіздігіне
қойылатын талаптар немесе сол немесе басқа балық консервілерінде
жасырылған қауіпті анықтау, балық өсірілетін аумаққа, жер асты суларының
және ауаның құрамына байланысты белгілі бір сортты балықты консервілеуге
жарамдылық критерийлерін әзірлеу. берілген көлдің, өзеннің немесе
жасанды су қоймасының, сондай-ақ балық консервілеуде қолданылатын тұздық
қызанақ, өсімдік майы сияқты көмекші материалдардың сақтау әдісі мен
құрамы, бұл ғылыми жаңалық және осы зерттеу жұмысының практикалық
құндылығы; бұл сайып келгенде Өзбекстанда тұщы су балық өндірісін
пайдалану картасын жасауға, консервілеу әдістеріне, балық консервілерін
өндіруге арналған консервілер ингредиенттерінің тізімін құруға,
өндірудің жеке технологиясын жасауға мүмкіндік береді.

{\bfseries Түйінді сөздер:} пресс-су балықтары, балық, күміс тұқы, тұқы,
табан балық, қауіпсіздік, белоктар, майлар, көмірсулар, витаминдер, ауыр
металдар.

\begin{center}
{\large\bfseries ИССЛЕДОВАНИЕ БЕЗОПАСНОСТИ ПРЕССНОВОДНЫХ РЫБ УЗБЕКИСТАНА И ИХ КОНСЕРВОВ}

{\bfseries И.Б. Пулатов\textsuperscript{1*},
К.М.Жураева\textsuperscript{2}, К.О.Додаев\textsuperscript{2},
Х.Н.Ниёзов\textsuperscript{2}}

\textsuperscript{1}Университет ветеринарной медицины, животноводчества и
биотехнологий,

Самарканд, Узбекистан, \textsuperscript{2}Химико-технологический
институт, Ташкент, Узбекистан,

\href{mailto:pulatov.i1990@mail.ru}{\ul{pulatov.i1990@mail.ru}}
\end{center}

Изложены способы промышленной переработки пресноводной рыбы, обеспечение
безопасности рыбных консервов, количественное и качественное определение
химического состава испытуемого ассортимента рыб и консервов из них, в
том числе белков с незаменимым аминокислотным составом, углеводов,
жиров, минеральных веществ, витаминов, ферментов, обнаружение опасных
компонентов, в том числе тяжелых металлов и их солей, продуктов
преобразования пестицидов, гербицидов, антибиотиков, наличие
радионуклидов с помощью методов газожидкостной хроматографии, способы их
устранения, изыскание путей их устранения, обеспечивая тем самым
требования пищевой безопасности того или иного сорта рыбы, либо
выявление опасности, таившейся в том или ином рыбном консерве,
разработка критериев пригодности для консервирования рыбы того или иного
сорта в зависимости от района выращивания рыбы, состава грунтовых вод и
воздуха данного озера, реки или искусственного водоема, а также способа
консервации и состава вспомогательных материалов, таких как соус
томатный, масло растительное, применяемых при консервировании рыбы, что
составляет научную новизну и практическую ценность данной
научно-исследовательской работы, в конечном итоге позволяющей создать
карту использования добычи пресноводной рыбы в Узбекистане, способы
консервирования, составление списка ингредиентов для консервов для
производства консервов из рыбы, разработка индивидуальной технологии для
производства.

{\bfseries Kлючевые слова:} прессноводная рыба, вобла, толстолоб, сазан,
сом, безопасность, белки, жиры, углеводы, витамины, тяжёлые металлы.

\begin{multicols}{2}
{\bfseries Introduction.} The work is devoted to the preparation of fish
feed from raw materials available in Uzbekistan and the processing of
freshwater fish from lakes, rivers and artificial reservoirs for canned
food. Studies show that feed produced from local raw materials contains
protein in the range of 16-22\%, and therefore the issue of bringing the
amount of protein in feed to 32\% is relevant. The production technology
is complete

The conservation of presswater fish has its own specific problems. Fish
in press water is contaminated with various toxic substances, depending
on its habitat, as a result, the content of certain harmful substances
in canned food often exceeds SanPIN and MPC standards, and there is no
widespread practice of canning freshwater fish. In this regard, there is
a need to identify the degree of harmfulness of freshwater fish,
depending on the habitat.

The chemical compositions of fish from various reservoirs of the
Republic of Uzbekistan, the chemical compositions of canned food made
from them were studied. Such studies were carried out for the first time
{[}1-3{]}.

{\bfseries Materials and methods.} The purpose of the work is to study the
ways of industrial processing of presswater fish, ensuring the safety of
canned fish products. To achieve this goal, it is necessary to solve the
following tasks:

- study of the chemical composition of fish;

- study of the chemical composition of canned food;

- comparison of the components of fish and canned fish with the maximum
allowable concentration of heavy metals;

- finding ways to reduce heavy metals in canned food.

An atomic absorption method was used for the determination of toxic
elements: lead, cadmium, copper, zinc, and iron. The technique was
developed by the Institute of Nutrition of the Russian Academy of
Medical Sciences, introduced by the State Standard of Russia {[}4-10{]}.
There is a detailed description and justification of the methodology
used for the quantitative and qualitative determination of heavy
metals);

The method of protein determination was also used. The methodology was
developed by the Federal State Budgetary Scientific Institution
``All-Russian Research Institute of the Meat Industry named after
V.M.Gorbatov (FGBNU VNIIMP named after V.M. Gorbatov) {[}11{]}.

{\bfseries Results and Discussion.}The chemical composition of fish meat is
characterized mainly by the content of water, fat, nitrogenous and
mineral substances, carbohydrates, enzymes, vitamins, etc.

The total amount of all protein substances in fish meat is, on average,
about 16\% (from 12 to 22\%). This includes salt-soluble proteins such
as globulins (myosin, actin, actomyosin, tropoliosin), water-soluble
proteins such as albumins (myogen, myoalbumin, globulin, myoprotein).
Myostromins, as well as nucleoproteins (histones, deoxyribose, purine
and pyrimidine bases) have been identified. Fish meat proteins are
complete, they contain all the essential amino acids in a well-balanced
ratio for consumption {[}12-13{]}.

At the same time, the heterocyclic amino acid histidine, when fish is
spoiled, turns into histamine, which has the properties of a synergistic
toxin in high doses. The stromal protein collagen is defective, but when
boiled in water, it turns into glue or glutin, which explains some
stickiness (stickiness) of boiled fresh fish meat, as well as gelation
of fish broths, which is important in the preparation of fish dishes.
Non-protein nitrogenous extractive substances (nitrogenous bases, amino
acids, acid amides, derivatives of guanidine, imidazole, purine, etc.),
despite the small content in meat (from 0.3 to 0.6\% in the meat of
sharks and rays up to 2.2\% ) give the fish a specific taste, smell and
affect the secretion of digestive juices in humans, stimulating appetite
and promoting better absorption of food. In this regard, the ear is a
more nutritious food product than the broth from the meat of
warm-blooded animals.

The fresh meat of some sea and ocean fish contains a specific substance
- trimethylamine oxide (TMAO), which has a pleasant smell (the smell of
fresh cucumber). During storage, TMAO turns into trimethylamine, which
has an unpleasant ammonia odor.

Fish oil has a lower melting point compared to the fat of warm-blooded
animals, which has a positive effect on its digestibility by the human
body. However, due to the significant amount of unsaturated fatty acids,
fish oil is easily subjected to oxidative deterioration due to the
contact of fat with atmospheric oxygen.

The fat content in fish meat is from 0.5 to 33\% and depends on the type
of fish, so they are conventionally divided into three groups: lean, in
which the fat content in the body does not exceed 4\% (cod, pike, pike),
medium fat - from 4 to 8\% fat (most carp fish, catfish, flounder) and
fatty - the amount of fat in the body is more than 8\% (sturgeon,
salmon, herring, etc.).

Fat is deposited in different parts of the fish: in sturgeon - between
muscle tissue, in cod - in the liver, in salmon - in the abdominal part,
in herring - under the skin, etc.

Carbohydrates in fish tissues, mainly in the muscles of the trunk and
liver, are mainly represented by glycogen (animal starch) and its
hydrolysis products (glucose, pyruvic and lactic acids). Their content
is from 0.03 to 0.8\% and makes up the main part of nitrogen-free
extractive substances.

Fish (especially in liver fat, caviar, internal fat) contain a
significant amount of fat-soluble vitamins A, D and vitamin E.

Vitamins of group B (B\textsubscript{1}, B\textsubscript{2},
B\textsubscript{3}, B\textsubscript{5}, B\textsubscript{6},
B\textsubscript{12}) in fish meat are approximately the same as in the
meat of warm-blooded animals.

Of the minerals, fish meat contains: potassium, sodium, magnesium,
chlorine, sulfur, phosphorus, iron, and other elements (from 0.9 to
1.6\% in total).

Particularly important is the content of the trace element iodine, which
is very small in other foods. For example, cod meat contains 800-2440
times more iodine than beef.

Water in fish meat - 55-83\%. The fatter the fish, the less water in its
tissues. So, in the meat of eel it is about 55\%, and in the meat of
perch and cod - up to 80\%.

Fish meat during heat treatment loses less water than the meat of
slaughtered animals and birds, so it tastes juicier. However, water
promotes the development of microorganisms, and also activates the
processes of protein and fat hydrolysis.

For the production of canned food from freshwater fish, it is necessary
to choose those fish whose habitat is not conducive to the accumulation
of heavy metals in their bodies.

Vobla is a fairly well-known and popular type of fish from the carp
family, despite its rather limited distribution. Canned vobla is no less
valuable than condensed milk or stew. To this day, this representative
of waterfowl is in demand.

Vobla grows up to 30 cm, but this is not the limit. There are
representatives up to 40 cm. The average weight is from 600 to 700 g,
the largest can weigh from 800 g to 1 kg. The body is flattened, but the
sides remain wide. At the top of the back, there seems to be a small
hump, and the back of the roach is even. The scales are smaller and fit
tighter. It is dark on the top of the back, even sometimes it seems that
it is black. Below is the transition to silver.

For experimental reproduction of the conservation of freshwater fish, in
particular the roach of the Aidar-Arnasay system of lakes in the Jizzakh
region, an analysis of the chemical composition of the fish was carried
out, the results are entered in table 1.
\end{multicols}

\begin{longtable}[]{@{}
  >{\raggedright\arraybackslash}p{(\columnwidth - 8\tabcolsep) * \real{0.0416}}
  >{\raggedright\arraybackslash}p{(\columnwidth - 8\tabcolsep) * \real{0.3827}}
  >{\raggedright\arraybackslash}p{(\columnwidth - 8\tabcolsep) * \real{0.1524}}
  >{\raggedright\arraybackslash}p{(\columnwidth - 8\tabcolsep) * \real{0.1713}}
  >{\raggedright\arraybackslash}p{(\columnwidth - 8\tabcolsep) * \real{0.2520}}@{}}
\caption*{Table 1. The chemical composition of fish meat of the roach of the
Aidar-Arnasay system lakes of Jizzakh region} \\
\toprule\noalign{}
№ & Name of indicator & Availability rate & Test results & Availability rate to metod \\
\midrule\noalign{}
\endhead
\bottomrule\noalign{}
\endlastfoot
1 & Mass fraction of protein, \% & 18,0 & 17,33 & GОSТ 13496.4-2019 \\
\hline
2 & Mass fraction of fate, \% & 2,8 & 21,3 & GОSТ 26829-6 п.2 \\
\hline
3 & Mass fraction of cation Мg\textsuperscript{2+}, \% & 2,5 & 2,93 &
\multirow{4}{*}{SF XII} \\
\cline{1-4}
4 & Mass fraction катиоан Na\textsuperscript{+} ,\% & 6,0 & 1,31 \\
\cline{1-4}
5 & Mass fraction of cation К\textsuperscript{+}, \% & 1,6 & 1,70 \\
\cline{1-4}
6 & Mass fraction of cation, Са\textsuperscript{2+}, \% & 4,0 & 2,05 \\
\hline
7 & Mass fraction Hg, ppm & 0,3 & Not detected & ГОСТ 26927 \\
\hline
8 & Mass fraction As, ppm no more & 1,0 & 0,02 & ГОСТ 26930 \\
\hline
9 & Mass fraction Pb, ppm no more & 0,2 & Not detected & ГОСТ 26932 \\
\hline
10 & Mass fraction Zn, ppm no more & 40,0 & 0,005 & ГОСТ 30178 \\
\hline
11 & Mass fraction Fe, ppm no more & 0,8 & 0,006 & \\
\hline
12 & Mass fraction Ni, ppm no more & 0,6 & Not detected & ГОСТ 27236 \\
\hline
\multirow{8}{*}{13} & Mass fraction vitamin group В: & & - &
\multirow{8}{*}{SF XII} \\
\cline{2-4}
& В\textsubscript{1,} \emph{мг}/100 g & - & - \\
\cline{2-4}
& В\textsubscript{2}, \emph{мг}/100 g & - & 0,07 \\
\cline{2-4}
& В\textsubscript{3,} \emph{мг}/100 g & - & - \\
\cline{2-4}
& В\textsubscript{5,} \emph{мг}/100 g & - & - \\
\cline{2-4}
& В\textsubscript{6,} \emph{мг}/100 g & - & - \\
\cline{2-4}
& В\textsubscript{9,} \emph{мг}/100 g & - & - \\
\cline{2-4}
& В\textsubscript{12,} \emph{мг}/100 \emph{g} & - & 0,12 \\
\end{longtable}

As can be seen from Table 1, the proportion of hazardous minerals, such
as arsenic, lead, mercury, zinc, nickel, in vobla meat is far from the
norm or absent.


\begin{longtable}[]{@{}
  >{\raggedright\arraybackslash}p{(\columnwidth - 8\tabcolsep) * \real{0.0391}}
  >{\raggedright\arraybackslash}p{(\columnwidth - 8\tabcolsep) * \real{0.4073}}
  >{\raggedright\arraybackslash}p{(\columnwidth - 8\tabcolsep) * \real{0.1274}}
  >{\raggedright\arraybackslash}p{(\columnwidth - 8\tabcolsep) * \real{0.1898}}
  >{\raggedright\arraybackslash}p{(\columnwidth - 8\tabcolsep) * \real{0.2364}}@{}}
\caption*{Таble 2. The chemical composition of canned fish from roach of the
Aidar-Arnasay system of lakes in the Jizzakh region} \\
\toprule\noalign{}
№ & Наименование показателя & Availability rate & Test results & Availability rate to metod \\
\midrule\noalign{}
\endhead
\bottomrule\noalign{}
\endlastfoot
1 & Mass fraction of protein, \% & 18 & 14,4 & GОSТ 7636-85 \\
\hline
2 & Mass fraction of fate, \% & 2,8 & 4,1 & GОSТ 7636-85 \\
\hline
3 & Mass fraction of cation Мg\textsuperscript{2+},\% & 2,5 & 466 &
\multirow{5}{*}{GОSТ EN 14084.2014} \\
\cline{1-4}
4 & Mass fraction катиоан Na\textsuperscript{+} ,\% & 6,0 & 1826 \\
\cline{1-4}
5 & Mass fraction of cation К\textsuperscript{+}, \% & 1,6 & 2401,7 \\
\cline{1-4}
6 & Mass fraction of cation, Са\textsuperscript{2+}, \% & 4,0 & 279,2 \\
\cline{1-4}
7 & Mass fraction Hg, ppm & 0,3 & Not detected \\
\hline
8 & Mass fraction As, ppm no more & 1,0 & Not detected & GОSТ ISO
8070/IDF119 \\
\hline
9 & Mass fraction Pb, ppm no more & 0,2 & Not detected & GOSТ EN
14084.2014 \\
\hline
10 & Mass fraction Zn, ppm no more & 40,0 & Not detected &
\multirow{3}{*}{GОSТ EN 14084.2014} \\
\cline{1-4}
11 & Mass fraction Fe, ppm no more & 0,8 & Not detected \\
\cline{1-4}
12 & Mass fraction Ni, ppm no more & 0,6 & Not detected \\
\end{longtable}

Table 2 shows that the proportion of hazardous minerals such as arsenic,
lead, mercury, zinc, nickel is much less than the standard, or absent
altogether. Moreover, B vitamins appear: B\textsubscript{1}=0.02;
B\textsubscript{2}=0.09; B6 0.01; B\textsubscript{12}
\textbackslash u003d 0.08 mg / 100 g.

The conclusion is that it is possible to produce canned fish in oil and
tomato sauce from the roach of the Aidar-Arnasay system of lakes in the
Jizzakh region. The main safety criteria for canned fish from these fish
have been verified, the MPCs of heavy metals in them comply with SanPIN
standards

The chemical composition of fish meat from the Chinaz district of the
Tashkent region is characterized by the content of water, fat,
nitrogenous and mineral substances, enzymes, vitamins, etc. som and the
results are entered in table 3.

\begin{longtable}[]{@{}
  >{\raggedright\arraybackslash}p{(\columnwidth - 8\tabcolsep) * \real{0.0461}}
  >{\raggedright\arraybackslash}p{(\columnwidth - 8\tabcolsep) * \real{0.4917}}
  >{\raggedright\arraybackslash}p{(\columnwidth - 8\tabcolsep) * \real{0.1376}}
  >{\raggedright\arraybackslash}p{(\columnwidth - 8\tabcolsep) * \real{0.1869}}
  >{\raggedright\arraybackslash}p{(\columnwidth - 8\tabcolsep) * \real{0.1378}}@{}}
\caption*{Таble 3. The chemical composition of fish meat in the Chinaz district of
the Tashkent region} \\
\toprule\noalign{}
№ & Name of indicator & Type of fish \\
& & \begin{minipage}[b]{\linewidth}\raggedright
Carp
\end{minipage} & \begin{minipage}[b]{\linewidth}\raggedright
Silver carp
\end{minipage} & \begin{minipage}[b]{\linewidth}\raggedright
Catfish
\end{minipage} \\
\midrule\noalign{}
\endhead
\bottomrule\noalign{}
\endlastfoot
1 & Mass fraction of protein, \% & 20,2 & 25,9 & 18,8 \\
\hline
2 & Mass fraction of fate, \% & 12,7 & 19,6 & 17,7 \\
\hline
3 & Mass fraction of cation Мg\textsuperscript{2+}, \% & 2,58 & 5,97 & 1,33 \\
\hline
4 & Mass fraction катиоан Na\textsuperscript{+} ,\% & 1,08 & 1,40 & 1,19 \\
\hline
5 & Mass fraction of cation К\textsuperscript{+}, \% & 1,40 & 1,23 & 1,70 \\
\hline
6 & Mass fraction of cation, Са\textsuperscript{2+}, \% & 0,29 & 3,12 & 0,36 \\
\hline
\multirow{8}{*}{7} & Mass fraction of vitamin group В: & & & \\
& В\textsubscript{1,} \emph{мг}/100 \emph{г} & 0,11 & 0,08 & 0,17 \\
& В\textsubscript{2}, \emph{мг}/100 \emph{г} & 0,09 & 0,14 & 0,11 \\
& В\textsubscript{3,} \emph{мг}/100 \emph{г} & 5,8 & 3,7 & 4,4 \\
& В\textsubscript{5,} \emph{мг}/100 \emph{г} & - & - & - \\
& В\textsubscript{6,} \emph{мг}/100 \emph{г} & - & 0,1 & - \\
& В\textsubscript{9,} \emph{мг}/100 \emph{г} & - & - & - \\
& В\textsubscript{12,} \emph{мг}/100 \emph{г} & 31,1 & 50,7 & 21,5 \\
\end{longtable}

Studies have also been carried out to determine the amount of heavy
metals and their salts, the presence of pesticides and herbicides,
antibiotics and phytohormones used in agriculture, radioactive
substances and radionuclides {[}14-15{]}.

In addition to organoleptic indicators, we checked the presence of heavy
metals in canned food, the source of which is the fish of the selected
area, the results are included in table 4.

\begin{longtable}[]{@{}
  >{\raggedright\arraybackslash}p{(\columnwidth - 12\tabcolsep) * \real{0.0469}}
  >{\raggedright\arraybackslash}p{(\columnwidth - 12\tabcolsep) * \real{0.3303}}
  >{\raggedright\arraybackslash}p{(\columnwidth - 12\tabcolsep) * \real{0.0849}}
  >{\raggedright\arraybackslash}p{(\columnwidth - 12\tabcolsep) * \real{0.1536}}
  >{\raggedright\arraybackslash}p{(\columnwidth - 12\tabcolsep) * \real{0.0922}}
  >{\raggedright\arraybackslash}p{(\columnwidth - 12\tabcolsep) * \real{0.0921}}
  >{\raggedright\arraybackslash}p{(\columnwidth - 12\tabcolsep) * \real{0.2000}}@{}}
\caption*{Tаble 4. The presence of heavy metals in canned freshwater fish of the
Chinaz district of the Tashkent region} \\
\toprule\noalign{}
\multirow{2}{*}{\begin{minipage}[b]{\linewidth}\raggedright
№
\end{minipage}} &
\multirow{2}{*}{\begin{minipage}[b]{\linewidth}\raggedright
Name of indicator
\end{minipage}} &
\multicolumn{3}{c}{
Availability rate} &
\multirow{2}{*}{\begin{minipage}[b]{\linewidth}\raggedright
MPC
\end{minipage}} &
\multirow{2}{*}{\begin{minipage}[b]{\linewidth}\raggedright
Availability rate to metod
\end{minipage}} \\
& & \begin{minipage}[b]{\linewidth}\raggedright
{\bfseries Сазан}
\end{minipage} & \begin{minipage}[b]{\linewidth}\raggedright
{\bfseries Толстолоб}
\end{minipage} & \begin{minipage}[b]{\linewidth}\raggedright
{\bfseries Сом}
\end{minipage} \\
\midrule\noalign{}
\endhead
\bottomrule\noalign{}
\endlastfoot
1 & Mass fraction Hg, ppm & 0,31 & 0,33 & 0,46 & 0,3 & GОSТ 26927 \\
2 & Mass fraction As, ppm & 0,02 & 0,05 & 0,04 & 1,0 & GОSТ 26930 \\
3 & Mass fraction Pb, ppm & 0,02 & 0,02 & 0,08 & 0,2 & GОSТ 26932 \\
4 & Mass fraction Zn, ppm & 2,07 & 1,78 & 2,47 & 40,0 & GОSТ 30178 \\
5 & Mass fraction Fe, ppm & 8,63 & 5,44 & 12,08 & - & \\
6 & Mass fraction Ni, ppm &
\multicolumn{3}{c}{Not detected} & - & GОSТ 27236 \\
\end{longtable}

\begin{multicols}{2}
Table 4 shows that in canned carp, silver carp and catfish grown in the
Chinaz district of the Tashkent region, the mercury content exceeds the
MPC, the worst reading in canned catfish. The conclusion is that catfish
is not suitable for the production of canned food from them.

Studies have also been carried out to determine the amount of heavy
metals and their salts, the presence of pesticides and herbicides,
antibiotics and phytohormones used in agriculture, radioactive
substances and radionuclides {[}16{]}.

The total amount of all protein substances in vobla meat is, on average,
about 18\%. This includes salt-soluble proteins such as globulins
(myosin, actin, actomyosin, tropoliosin), water-soluble proteins such as
albumins (myogen, myoalbumin, globulin, myoprotein). Myostromins, as
well as nucleoproteins (histones, deoxyribose, purine and pyrimidine
bases) have been identified. Fish meat proteins are complete, they
contain all the essential amino acids in a well-balanced ratio for
consumption {[}14-16{]}.

At the same time, the heterocyclic amino acid histidine, when fish is
spoiled, turns into histamine, which has the properties of a synergistic
toxin in high doses.

The stromal protein collagen is defective, but when boiled in water, it
turns into glue or glutin, which explains some stickiness (stickiness)
of boiled fresh fish meat, as well as gelation of fish broths, which is
important in the preparation of fish dishes.

Non-protein nitrogenous extractive substances (nitrogenous bases, amino
acids, acid amides, derivatives of guanidine, imidazole, purine, etc.),
despite the small content in meat (from 0.3 to 0.6\% in the meat of
sharks and rays up to 2.2\% ) give the fish a specific taste, smell and
affect the secretion of digestive juices in humans, stimulating appetite
and promoting better absorption of food. In this regard, the ear is a
more nutritious food product than the broth from the meat of
warm-blooded animals.

The fat content in fish meat is from 0.5 to 33\% and depends on the type
of fish, so they are conventionally divided into three groups: lean, in
which the fat content in the body does not exceed 4\% (cod, pike, pike),
medium fat - from 4 to 8\% fat (most carp fish, catfish, flounder) and
fatty - the amount of fat in the body is more than 8\% (sturgeon,
salmon, herring, etc.).

Fish (especially in liver fat, caviar, internal fat) contain a
significant amount of fat-soluble vitamins A, D and vitamin E.

There are about the same amount of B vitamins (B\textsubscript{1},
B\textsubscript{2}, B\textsubscript{3}, B\textsubscript{5},
B\textsubscript{6}, B\textsubscript{12}) in fish meat as in the meat of
warm-blooded animals.

Of the minerals, fish meat contains: potassium, sodium, magnesium,
chlorine, sulfur, phosphorus, iron, and other elements (from 0.9 to
1.6\% in total).

Particularly important is the content of the trace element iodine, which
is very small in other foods. For example, cod meat contains 800-2440
times more iodine than beef.

Water in fish meat - 55-83\%. The fatter the fish, the less water in its
tissues. So, in the meat of eel it is about 55\%, and in the meat of
perch and cod - up to 80\%.

Fish meat during heat treatment loses less water than the meat of
slaughtered animals and birds, so it tastes juicier. However, water
promotes the development of microorganisms, and also activates the
processes of protein and fat hydrolysis

{\bfseries Conclusions.} Thus, the proportion of hazardous minerals, such
as arsenic, lead, mercury, zinc, nickel, iron, in vobla meat is much
lower than the permissible standard or they are absent {[}10-14{]}. This
means that the roach living in the Aidar-Arnasay system of lakes in the
Jizzakh region can be consumed both fresh and canned. Vitamins of group
B appear in canned voble: B\textsubscript{1}=0.02;
B\textsubscript{2}=0.09; B\textsubscript{6} =0.01; B\textsubscript{12}
\textbackslash u003d 0.08 mg / 100 g.

Also, studies were carried out to determine the amount of heavy metals
and their salts, the presence of pesticides and herbicides, antibiotics
and phytohormones used in agriculture, radioactive substances and
radionuclides in three types of fish in the Chinaz district of the
Tashkent region: carp, silver carp, catfish. These results coincide with
the studies previously carried out by the authors of the work I.Pulatov
and K.Dodaev, which are carried out according to GOSTs {[}1{]}.

In canned carp, silver carp and catfish grown in the Chinaz district of
the Tashkent region, the mercury content exceeds the MPC, the worst
reading in canned catfish. This is confirmed by the restrictions given
in GOSTs {[}10-14{]}. The conclusion is that catfish is not suitable for
the production of canned food from them.

Studies have also been carried out to determine the amount of heavy
metals and their salts, the presence of pesticides and herbicides,
antibiotics and phytohormones used in agriculture, radioactive
substances and radionuclides in canned food of their selected three
varieties of fish {[}2-3{]}.

Fish from lakes, rivers and artificial reservoirs have been studied for
the presence of heavy metals in their meat, such as arsenic, lead,
mercury, zinc, nickel, and iron. Canned food was made from them, the
presence of these metals and their compounds was investigated in them,
and conclusions were drawn that it is possible to make canned food
depending on the habitat and type of fish.

Freshwater fish habitats can be sources of radioactive substances,
pesticides, herbicides, antibiotics, which is important when solving the
problem of using or not using fish for canning.

These conclusions are substantiated by studies of the chemical
composition of fish, canned food made from them, comparative analyzes of
the components of fish and canned fish with the maximum allowable
concentration of heavy metals, the use of certain methods that reduce
the content of toxins in canned food to acceptable limits and below,
thereby ensuring the safety of canned food.
\end{multicols}

{\bfseries References}

\begin{enumerate}
\item
Pulatov I.B., Dodaev K.O. The results of the study of canned
presswater fish in tomato sauce // Universum Technical Sciences. Moscow.
No. 6 (4-99), 2022. -P.22-25.

\item
GOST 30178-96 (Interstate standard). Raw materials and food products.
Atomic absorption method for the determination of toxic elements.
Moscow, Standartinform, 2010.

\item
GOST 25011-2017 (interstate standard). Meat and meat products.
Protein determination methods. Moscow, Standartinform, 2018.

\item
Vladimtseva T.M. Technology of fish and fish products. Methods for
determining the quality of fish products. Krasnoyarsk. 2019. -105 p.

\item
Modern problems of food quality and safety in the light of the
requirements of the technical regulations of the customs union /
Proceedings of the international scientific and technical conference.
Krasnodar 2014.

\item
Vavilova N.I. Commodity research and examination of fish products and
seafood. Saratov 2017. -52 p.

\item
Volkov A.Kh., Papunidi E.K. Yakupova L.F. Assessment of the quality
and safety of fish and seafood. Tutorial. Kazan, 2020. -154 p.

\item
Normakhmatov R., Pulatov I. The mass composition of fish is an
important commodity and technological indicator. Argo-ilm magazine, No1,
Tashkent - 2020, - P. 64-65.

\item
GОSТ 26927. Mercury determination method.

\item
GОSТ 26930. Arsenic determination method.

\item
GОSТ 26932. Lead determination method

\item
GОSТ 27236. Nickel determination method

\item
GОSТ 30178. Method for determination of zinc and iron.

\item
GОSТ 13865-2000. Canned natural fish with added oil.

\item
GOST 16978 - 99. Canned fish in tomato sauce. Specifications.

\item
Mamontov Yu.P. Current state and prospects for the development of
aquaculture in Russia. - Abstract. dis. d. - Krasnodar: KSAU, 2000. - 40
p.
\end{enumerate}

\emph{{\bfseries Information about the authors}}

\begin{itemize}
\item
Pulatov I.B. University of Veterinary Medicine, Animal Husbandry and
Biotechnology, Samarkand, Uzbekistan,

\item
pulatov.i1990@mail.ru;

\item
Zhuraeva K.M., -magstrant, , Tashkent Institute of Chemical Technology,
Tashkent, Uzbekistan, kamolaxonjurayeva193@gmail.com

\item
Dodaev K.O. - Doctor of Technical Sciences prof. ,Tashkent Institute of
Chemical Technology, Tashkent, Uzbekistan., Dodoev@rambler.ru,

\item
Niyozov Kh.N. doctoral student, xusan, Tashkent Institute of Chemical
Technology, Tashkent, Uzbekistan., niyozov.90@mail.ru,
\end{itemize}

\emph{{\bfseries Сведения об авторах:}}

\begin{itemize}
\item
Пулатов И.Б. Университет ветеринарной медицины, животноводчества и
биотехнологий, Самарканд, Узбекистан,
\href{mailto:pulatov.i1990@mail.ru}{\ul{pulatov.i1990@mail.ru}};

\item
Жураева К.М., -магтстрант, Ташкентский химико-технологический институт,
Ташкент, Узбекистан,
\href{mailto:kamolaxonjurayeva193@gmail.com}{\ul{kamolaxonjurayeva193@gmail.com}}

\item
Додаев К.О.- д.т.н. проф., Ташкентский химико-технологический институт,
Ташкент, Узбекистан.,
\href{mailto:Dodoev@rambler.ru}{\nolinkurl{Dodoev@rambler.ru}}

\item
Ниёзов Х.Н. докторант, Ташкентский химико-технологический институт,
Ташкент, Узбекистан.,
\href{mailto:xusan.niyozov.90@mail.ru}{\ul{xusan.niyozov.90@mail.ru}}
\end{itemize}
