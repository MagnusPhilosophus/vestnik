\let\cleardoublepage\clearpage
\chapter{Экономика, бизнес и услуги}

{\bfseries SRSTI 06.81.65}

\section{MOTIVATION SYSTEM OF CIVIL SERVANTS OF KAZAKHSTAN: ASSESSMENT
AND PROSPECTS FOR IMPROVEMENT}

\begin{center}
{\bfseries Z.R. Karbetova\textsuperscript{1*},
A.S.Baktymbet\textsuperscript{1}, Sh.R. Karbetova\textsuperscript{2}}

\emph{{\bfseries \textsuperscript{1}}}Kazakh University of Technology and
Business, Astana, Kazakhstan

\emph{{\bfseries \textsuperscript{2}}}International Transport and
Humanitarian University, Almaty, Kazakhstan

e-mаil \href{mailto:kzr_2011@mail.ru}{\nolinkurl{kzr\_2011@mail.ru}}
\end{center}

The success of public administration depends on the extent to which
civil servants realize their professional potential. The insufficient
level of motivation of civil servants is an obstacle to their effective
work. Ways to increase professional potential through motivational
interventions aimed at developing the competence of civil servants
remain undeveloped. Therefore, the main focus of this study will be on
the assessment of the modern system of motivation of civil servants and
consideration of ways to improve it in the Republic of Kazakhstan, which
is relevant and of great scientific and practical importance.

The paper notes the features of the motivation of civil servants,
analyzes foreign experience in motivating the professional activities of
civil servants and considers the directions for stimulating their work;
a modern system for evaluating the performance of civil servants based
on a survey has been studied. In the process of conducting an empirical
study in Kazakhstan, the processing of questionnaires made it possible
to identify the main factors motivating the professional activities of
active civil servants. The sample set of the conducted survey allowed to
achieve a high level of representativeness of the results and
conclusions.

The new model of civil service, which has been tested, is based on a
factor-score scale. The new system of remuneration of civil servants
will make it possible to select the most conscientious and versatile
employees for the civil service, which will improve the quality of
public administration. Thus, we carried out an analysis of the
effectiveness of the system of motivation of civil servants in the
Republic of Kazakhstan, evaluated them, and developed measures to
improve the motivation for the professional activities of civil servants
of the Republic of Kazakhstan.

{\bfseries Keywords:} motivation systems, civil servants, assessment,
management, analysis, remuneration, efficiency.

\begin{center}
{\large\bfseries СИСТЕМА МОТИВАЦИИ ГОСУДАРСТВЕННЫХ СЛУЖАЩИХ КАЗАХСТАНА: ОЦЕНКА И
ПЕРСПЕКТИВЫ СОВЕРШЕНСТВОВАНИЯ}

{\bfseries З.Р.Карбетова\textsuperscript{1*},
А.С.Бактымбет\textsuperscript{1}, Ш.Р.Карбетова\textsuperscript{1}}

{\bfseries \textsuperscript{1}}Казахский университет технологии и бизнеса,
г.Астана, Казахстан

\emph{\textsuperscript{2}}Международный транспортно-гуманитарный
университет, г.Алматы, Казахстан

e-mаil \href{mailto:kzr_2011@mail.ru}{\nolinkurl{kzr\_2011@mail.ru}}
\end{center}

Успех деятельности государственного управления зависит от того, в какой
степени государственные служащие реализовывают свой профессиональный
потенциал. Недостаточный уровень мотивации госслужащих является
препятствием для их эффективной деятельности. Неразработанными остаются
пути повышения профессионального потенциала посредством мотивационного
вмешательства, направленного на развитие компетентности государственных
служащих. Поэтому основной акцент данного исследования будет сделан на
оценке современной системы мотивации государственных служащих~ и
рассмотрении пути ее совершенствования в Республике Казахстан, что
является актуальным и имеет важное научное и практическое значение.

В работе отмечены особенности мотивации государственных служащих,
проанализи-

рован зарубежный опыт мотивации профессиональной деятельности
государственных служащих и рассмотрены направления стимулированияих
труда; исследована современная система оценки деятельности
государственных служащих на основе анкетирования. В процессе проведения
эмпирического исследования в Казахстане, обработка анкет позволила
выявить основные факторы мотивации профессиональной деятельности
действующих государственных служащих. Выборочная совокупность
проведенного анкетирования позволила достичь высокого уровня репрезентативности результатов и выводов.

Прошедшая апробация, новаямодель государственной службы основана на
факторно - бальной шкале. Новая система оплаты труда госслужащих
позволит отобрать на государственную службу наиболее добросовестных и
разносторонне грамотных сотрудников, что повысит качество
государственного управления. Таким образом, нами был осуществлён анализ
эффективности системы мотивации государственных служащих в РК, проведена
их оценка, иразработаны мероприятия по совершенствованию мотивации
профессиональной деятельности государственных служащих Республики
Казахстан.

{\bfseries Ключевые слова:} системы мотивации, государственные служащие,
оценка, управ-ление, анализ, оплата труда, эффективность.

\begin{center}
{\large\bfseries ҚАЗАҚСТАН РЕСПУБЛИКАСЫНДАҒЫ МЕМЛЕКЕТТІК}

{\bfseries ҚЫЗМЕТШІЛЕРІН МОТИВАЦИЯЛАУ ЖҮЙЕСІ: БАҒАЛАУ ЖӘНЕ ЖЕТІЛДІРУ
БОЛАШАҒЫ}

{\bfseries З.Р.Карбетова\textsuperscript{1*},
А.С}.{\bfseries Бақтымбет\textsuperscript{1}},
{\bfseries Ш.Р.Карбетова\textsuperscript{2}}

{\bfseries \textsuperscript{1}}Қазақ технология және бизнес университеті,
Астана к., Қазақстан

{\bfseries \textsuperscript{2}}Халықаралық
көлікжәнегуманитарлықуниверситеті, Алматы к., Қазақстан

e-mаil \href{mailto:kzr_2011@mail.ru}{\nolinkurl{kzr\_2011@mail.ru}}
\end{center}

Мемлекеттік басқарудың жетістікті болуы мемлекеттік қызметшілердің
кәсіби әлеуетін қандай дәрежеде іске асыруына байланысты. Мемлекеттік
қызметшілерді ынталандыру деңгейінің жеткіліксіздігі олардың нәтижелі
жұмыс істеуіне кедергі болып отыр. Мемлекеттік қызметшілердің
құзыреттілігін дамытуға бағытталған уәждемелік ықпал ету арқылы кәсіби
әлеуетті арттыру жолдары әлі де дамымаған күйде қалып отыр. Сондықтан
бұл зерттеудің негізгі бағыты өзекті және ғылыми-тәжірибелік маңызы зор
мемлекеттік қызметшілерді ынталандырудың заманауи жүйесін бағалау және
оны Қазақстан Республикасында жетілдіру жолдарын қарастыру болмақ.

Сондықтан мемлекеттік қызметшілерді ынталандырудың заманауи жүйесін
бағалау және оны Қазақстан Республикасында жетілдіру жолдарын қарастыру
бұл зерттеудің негізгі бағыты өзекті және ғылыми-тәжірибелік маңызы зор
болып келеді.

Жұмыста мемлекеттік қызметшілерді ынталандыру ерекшеліктері атап
көрсетіліп, мемлекеттік қызметшілердің кәсіби қызметін ынталандырудың
шетелдік тәжірибесі талданып және олардың жұмысын ынталандыру бағыттары
қарастырылды; сауалнама негізінде мемлекеттік қызметшілердің қызметін
бағалаудың заманауи жүйесі зерделенді. Қазақстанда эмпирикалық зерттеу
жүргізу барысында сауалнамаларды өңдеу белсенді мемлекеттік
қызметшілердің кәсіби қызметін ынталандыратын негізгі факторларды
анықтауға мүмкіндік берді. Өткізілген сауалнаманың іріктеме жинағы
нәтижелер мен қорытындылардың репрезентативтілігінің жоғары деңгейіне
қол жеткізуге мүмкіндік берді.

Тестілеуден өткен мемлекеттік қызметтің жаңа моделі факторлық-баллдық
жүйеге сүйенеді. Мемлекеттік қызметшілерге еңбекақы төлеудің жаңа жүйесі
мемлекеттік қызметке ең адал және жан-жақты сауатты қызметкерлерді
іріктеуге мүмкіндік береді, бұл мемлекеттік басқарудың сапасын
арттырады. Осылайша, біз Қазақстан Республикасындағы мемлекеттік
қызметшілерді ынталандыру жүйесінің тиімділігіне талдау жүргізіп, оларды
бағалап, Қазақстан Республикасы мемлекеттік қызметшілерінің кәсіби
қызметіне ынталандыруды жетілдіру бойынша шаралар әзірледік.

{\bfseries Түйін сөздер:} мотивация жүйелері, мемлекеттік қызметшілер,
бағалау, басқару, талдау, еңбекақы, тиімділік.

\begin{multicols}{2}
The issues of formation of motivation for the professional growth of
civil servants have a high degree of relevance, and are determined by
the importance of the processes of improving the motivation for the
professional activity of civil servants. The insufficient level of
motivation of civil servants is an obstacle to their effective
performance. The effectiveness of the implementation of state
development programs that bring the country to new higher levels of
socio-economic development and prosperity depends on how efficiently and
professionally the professional duties of civil servants are performed
{[}1{]}.

The purpose of the work is to develop measures to enhance the motivation
of civil servants in Kazakhstan by examining motivational theories,
studying foreign experience, analyzing the current staffing and
structure of state bodies, exploring modern evaluation systems, and
proposing improvements.

The scientific novelty of the research results lies in the development
of a comprehensive methodological toolkit to motivate the professional
activities of civil servants. This can be achieved through the
integrated use of motivation mechanisms and personnel management.

{\bfseries Materials and methods.} The study was conducted using general
scientific methods of cognition: historical and logical approaches,
systemic and situational, questioning and methods of economic analysis.
In the course of this study, various general scientific and private
research methods were also used, including: methods of synthesis and
deduction, description and statistical methods, graphic-analytical and
calculation methods. The analysis of statistical data was carried out
using the methods of grouping, comparison and generalization. For the
collection and processing of empirical data, sociological methods of
collecting information were used: a questionnaire survey and analysis of
documents {[}3{]}.

{\bfseries Results and discussion.} In modern conditions, for solving
complex problems, issues of the quality of personnel management at all
levels of government are of paramount importance. The effectiveness of
the implementation of state development programs that bring the country
to new higher levels of socio-economic development and prosperity
depends on how effectively and professionally the professional duties of
civil servants are carried out. It should be noted that for the quality
work of the state apparatus, a high level of professional development of
the performers themselves, that is, civil servants, is necessary.

An analysis of the theoretical and methodological approach to the system
of motivation for the professional activities of civil servants showed
that it more reflects the effect of material factors of motivation. Of
particular relevance to the professional growth of civil servants are
various technologies and artificial intelligence that can perform the
typical duties of an official - from creating template letters and
processing large amounts of information to providing public services.

In this study, an attempt was made to develop new ways and methods of
organizing the motivation of civil servants and answer the most pressing
questions:

1. What is the most effective way to achieve high professional labor
returns from civil servants, and what motives drive them?

2. When are civil servants ready to show all their professional and
personal qualities to the maximum for the successful functioning of
their organization, and what will lead to an irresponsible and negligent
approach to work?

3. What are the factors that form the motivation of civil servants, and
what impact do they have on their professional activities?

Therefore, in order to determine the impact on the level of motivation
of a civil servant for professional growth, it is necessary to study
what the domestic apparatus of civil servants is like in general, what
are its main characteristics, and also what general trends in its
development are currently taking place. Therefore, the leadership of an
organization or government body should have a good knowledge of the
theoretical aspects of the problem under consideration in order to
properly organize the work of their subordinates and be able to direct
them to the effective performance of their functional duties.

In the course of the study, a wide range of various sources was studied
and the main schools and directions of motivation for the professional
activity of civil servants in the Republic of Kazakhstan were
identified{[}4,5{]}.

Source 4 studied the attitudes of 740 civil servants in Kazakhstan
towards their careers, organizational culture and climate at work. The
results of the study made it possible to highlight three main points
regarding the motivation and attitude of civil servants to their work in
Kazakhstan.

1. Three out of four respondents emphasize that primary and secondary
benefits are important to their motivation and performance. Interviewed
civil servants in Kazakhstan demonstrated high levels of ``public
service motivation'' (PSM) and intrinsic motivation compared to
extrinsic motivation.

2. Civil servants in Kazakhstan demonstrate a positive attitude towards
their colleagues and team spirit in the workplace. The high level of
intrinsic motivation and MSM of civil servants in Kazakhstan explain the
willingness to change and reform, despite the existing difficulties.

3. There is a widespread opinion among civil servants that promotion
does not depend on personal merit, and this important problem needs to
be addressed. The results obtained indicate the need to increase the
number of comparative studies of the motivation of civil servants in
Kazakhstan.

According to source 5, the main directions and approaches to assessing
the activities of civil servants as a system of motivation for their
professional activities are considered. The features and key elements of
the motivation of the professional activity of civil servants of the
most developed countries were identified on the basis of a study of
foreign experience.

In the process of conducting an empirical study in Kazakhstan, the
processing of questionnaires made it possible to identify the main
factors motivating the professional activities of active civil servants.
The sample set of the conducted survey allowed to achieve a high level
of representativeness of the results and conclusions.

The main objective of the study is to study those samples of foreign
experience that force the creation of a modern and efficient personnel
management system, taking into account the specifics of
Kazakhstan\textquotesingle s development and statehood. Analyzing
foreign experience {[}6{]}, it should be noted a number of features that
are of practical interest and can be taken into account in the process
of improving the civil service system of Kazakhstan:

- professionalization of the state apparatus on a permanent basis;

- flexible methods of human resource management (personnel management);

- career development management: flexible system of career advancement,
high career mobility, career advancement based on professional
achievements, i.e. not only depending on work experience;

- the results of the assessment of civil servants - the basis for career
growth and material rewards;

- availability of various employment schemes for the public service:

- permanent civil servants and employees on a contract basis;

- modern organization of work: more flexible working hours, introduction
of advanced technologies;

- modern methods of material incentives, including wages (taking into
account individual and collective results of work).

To date, the country has created its own model of public service, which
combines the achievements of world experience and the specifics of the
domestic public administration system. The new model consists of two
parts - constant and variable {[}7{]}. The permanent part is based on a
factor-point scale (FBS), according to which the amount of wages depends
on the nature, volume and complexity of the work. The variable part,
bonuses, will be accrued to effective employees based on the results of
their performance evaluation. The study revealed the results of the work
done in the framework of the professionalization of the state apparatus
in the Republic of Kazakhstan.

In Kazakhstan, the FBS of civil servants has been introduced, which
radically changes the system of remuneration. According to the Agency
for Civil Service Affairs, the main advantage of the FBS is the
avoidance of ``leveling''. According to the new system, the official
salary of civil servants will depend on the volume and complexity of
work, which will increase the responsibility and motivation of civil
servants.

According to the head of the department of the Agency of the Republic of
Kazakhstan for Civil Service Affairs, in connection with the
introduction of the FBS, those wishing to enter the civil service have
increased. According to analysts, the new system of remuneration of
civil servants will make it possible to select the most conscientious
and versatile employees for the civil service. Consequently, the quality
of public administration depends on civil servants, their work, which
reflects their professional level.

What is the apparatus of civil servants of Kazakhstan, what are its main
characteristics and general trends in its development are currently
presented below. As of January 1, 2020, the staff of civil servants
amounted to 97,403 units. At the same time, the actual number of civil
servants decreased by 1,927 {[}8, 9{]} people compared to 2018. The
release of positions occurred without the actual reduction of employees
by not filling vacancies, the saved funds were directed to increase the
wages of the most effective employees.

In recent years, there has been a constant trend of understaffing
throughout the entire structure of the state apparatus. According to the
results of 2020 data, staffing was 93.22\%. The dynamics of the analyzed
indicator during the analyzed period was positive. At the same time, it
should be noted that the persistence of a significant percentage of
understaffing for a sufficiently long time indicates the presence of
problems in the personnel service in terms of organizing planning and
forecasting personnel.

According to the National report on the state of the civil service in
the Republic of Kazakhstan, as of January 1, 2022, the number of civil
servants was 88,321. At the same time, the actual number of government
agencies decreased by 6.4\% {[}10{]}. In order to solve state problems
in the medium term, the further vector of development of the civil
service will be aimed at solving the issues of excessive bureaucracy of
the state apparatus, strengthening communication and interaction with
the population, attracting talented youth to the civil service,
automating personnel processes and reorienting personnel management
services to strengthen the personnel potential of state bodies.

The educational level of civil servants in the Republic of Kazakhstan is
quite high and tends to further increase. Scientists confirm the
influence to a large extent of such factors as the level of education,
economic status, age, etc., on the indicators of participation in
sociological studies on effective motivation and effective personnel
management {[}11{]}. In terms of the level of education, respondents
with higher education dominate the most - 92.69\%, including:
specialists or bachelors (84.44\%), as well as masters (82.5\%). These
results coincided with the data of the Agency for Civil Service Affairs
(92\%). This situation with the qualitative composition of public
authorities contributes to an increase in the level of services they
provide.

In our opinion, the constant increase in the level of education of
existing employees is also explained by the systematic implementation of
the 100 Concrete Steps Nation Plan, which made it possible to strengthen
the principles of meritocracy in the selection and career advancement
{[}12{]}. This improved the quality of the civil servants. Currently,
92\% of employees have higher education, while at the central level this
figure reaches 100\%. Persons with secondary and secondary vocational
education (8\%) occupy low-level positions of the district and rural
levels, about 2.8 thousand are graduates of foreign universities. The
number of graduates of the international scholarship "Bolashak" over the
past 3 years has increased by 12\% (from 505 in 2017 to 565 in 2019),
the Academy of Public Administration under the President - by 4\% (from
740 to 770).

Over the past few years, there has been a stability in the personnel of
the state apparatus. The outflow of personnel from the civil service
system is within 6\% and tends to gradually decrease (in 2015 - 11.2\%,
in 2016 - 6.3\%, in 2017 - 6.2\%, in 2018 - 6.2\% , in 2019 - 6\%).

This situation with the qualitative composition of public authorities
contributes to an increase in the level of services they provide, which
is noted in the Strategic Plan of the Agency of the Republic of
Kazakhstan for Civil Service Affairs and Anti-Corruption for 2018-2021:
``Over the years of assessment, starting from 2011, there has been an
increase in the average government agencies" {[}13{]}. At the same time,
the same document acknowledges the existence of serious problems with
the quality of services provided by state bodies today: ``There is a
trend of increasing the number of justified complaints received by state
bodies about the quality of public services.''

Most of all, citizens aged 30-39 years old (34.9\%) participated in the
survey, citizens aged 20-29 years old (23.9\%) and 40-49 years old
(23.2\%) took second place. This fact also confirms the assessment of
the Agency for Civil Service Affairs, according to which the average age
of civil servants is 39 years, including by age groups: up to 30 years
old - 23\%, from 30 to 40 years old - 34.1\%, from 40 to 50 years -
22.2\%, 50 years and older - 20.7\%72. This fact indicates the current
trend towards a gradual rejuvenation of the apparatus of state
administration, which increases the opportunities for training,
retraining and advanced training, as young employees more effectively
perceive new knowledge. It is also important that there is a gender
balance in the personnel structure of the civil service in Kazakhstan.
Thus, according to these data, the proportion of men is 40.2\%, and the
proportion of women is 59.8\%, which indicates that women are more
likely to participate in sociological research than men. The results of
the study confirm the information of the Agency for Civil Service
Affairs of the Republic of Kazakhstan on the proportion of women among
civil servants: 59.8\%.

In Kazakhstan, the proportion of civil servants to the economically
active population is approximately 1.1\%, and there are about 197.7
citizens per civil servant. At the same time, for example, in the OECD
countries, these figures are 2.43\% and 116 people, respectively.

At any enterprise and in any organization, including government bodies,
the following issues should be resolved first of all:

- providing employees with at least a sufficient level of wages;

- providing employees with an attractive social package;

- providing employees with a workplace equipped with everything
necessary for comfortable and high-quality work.

Unmotivated civil servants are not exclusively a problem for the Kazakh
state. Many problems of bureaucracy are relevant for all countries of
the world. Even such sociologists as M. Weber and R. Merton noted that
dysfunctional signs of bureaucracy are found both in the West and in the
East {[}14{]}.

Thus, the most serious motivating factors in choosing the profession of
a civil servant are the motives dictated by categories of a high order.
Recently, however, facts from the modern life of various countries, when
civil servants began to trample on state and public interests, speak of
a catastrophic decline in the morale of civil servants.

But explanations of these facts showed that A. Maslow\textquotesingle s
conclusions regarding basic human needs are true at any time and in any
state. As a result of the research, it turned out that the reasons for
this development of the situation are ``lower wages, insufficient
material and technical equipment for the performance of official duties,
an inadequate state budget and pressure to remain efficient while
reducing resources and costs'' {[}15{]}.

Of particular interest is the study of existing theories of motivation
in terms of the possibility of using their provisions, approaches and
postulates in motivating the professional activities of domestic civil
servants. In order to assess and identify the degree of correlation
between the elements of the motivation system and the efficiency /
performance of domestic civil servants, we conducted an empirical study
based on a survey of existing civil servants.

The choice of empirical assessment is justified by the fact that an
effective motivation system should be maximally integrated with the
assessment system. Using the current assessment system to identify the
motivational profile of civil servants is inappropriate. Therefore, a
survey of civil servants was conducted in order to identify the level of
motivation of their professional activities. The impersonality of this
survey allowed to increase the level of reliability of the study

The questionnaire was compiled in such a way as to be able to process
the results according to the scientifically based Methodology of K.
Zamfir in the modification of A. Rean "Studying the motivation of
professional activity" {[}16{]}. The need to conduct a survey of civil
servants is to identify the level of motivation of their professional
activities. These questionnaires were designed and compiled in such a
way as to give the most detailed and large-scale idea of
\hspace{0pt}\hspace{0pt}the effectiveness of the current motivation
system and its shortcomings.

The survey was conducted among active civil servants in two stages. At
the first stage of the empirical study, the goal was to collect data to
identify the degree of influence of external and internal factors. The
second stage of the empirical study was the questioning of the same
group of respondents, but within the framework of another survey. The
survey within this stage is aimed at clarifying and detailing the
results of the first stage. The survey results were processed and
summarized in a table 1.
\end{multicols}

\begin{longtable}[]{@{}
  >{\raggedright\arraybackslash}p{(\columnwidth - 10\tabcolsep) * \real{0.2386}}
  >{\raggedright\arraybackslash}p{(\columnwidth - 10\tabcolsep) * \real{0.1622}}
  >{\raggedright\arraybackslash}p{(\columnwidth - 10\tabcolsep) * \real{0.1474}}
  >{\raggedright\arraybackslash}p{(\columnwidth - 10\tabcolsep) * \real{0.1620}}
  >{\raggedright\arraybackslash}p{(\columnwidth - 10\tabcolsep) * \real{0.1475}}
  >{\raggedright\arraybackslash}p{(\columnwidth - 10\tabcolsep) * \real{0.1424}}@{}}
\caption*{Table 1. The opinion of civil servants on the level of factors affecting
the motivation of their work} \\
\toprule\noalign{}
Question &
\multicolumn{5}{>{\raggedright\arraybackslash}p{(\columnwidth - 10\tabcolsep) * \real{0.7614} + 8\tabcolsep}@{}}{%
Answer options} \\
\midrule\noalign{}
\endhead
\bottomrule\noalign{}
\endlastfoot
1 & 2 \\
\multirow{2}{=}{1). Would you agree to change jobs, subject to a
reduction in pay by 25\% - 30\%, but with more interesting
responsibilities?} &
Yes &
No &
Rather yes than no &
More likely no than yes &
Difficult to answer \\
& 18,5 & 33,1 & 17,9 & 19,3 & 11,2 \vspace{1.7cm}\\

\multirow{2}{=}{2). In the process of working in the public service, what will be the
most important for you?} &
Possibility of obtaining
knowledge and skills &
Recognition from peers &
Official height &
Management recognition &
Difficult to answer \\
& 18,1 & 13,2 & 27,2 & 35,4 & 6,2 \\

\multirow{2}{=}{3). What factors, from your point of view, have the
greatest impact on reducing the motivation and professionalism of civil
servants?} & Recruitment and promotion of personnel through
acquaintance, personal loyalty

news & Underestimation of the role of staff professionalism & Lack of
demand for honest and principled civil servants & Low level of
professional culture of managers & Other \\
& 45,5 & 21,8 & 16,1 & 11,6 & 5,0 \\

\multirow{2}{=}{4). What stimulates you to increase the level of
professional growth?} & High level of independence (freedom) &
Competition in the work team & Possibility of realization of acquired
knowledge & Careerprospect & Encouragement / recognition of merit by
management \\
& 21,1 & 7,6 & 25,9 & 32,2 & 14,2 \\
\multirow{2}{=}{5). How do you assess the current system of motivation
for the professional activities of civil servants?} & Effective &
Noteffective & Very little is known about it & Haven\textquotesingle t
come across it in
practice & Difficult to answer \\
& 32,2 & 27,6 & 16,5 & 17,0 & 6,7 \vspace{1cm}\\
\multirow{2}{=}{6). Which of the ways to increase professional growth do
you consider the most effective?} &
Obtaining additional professional education & Scientific research activities
& Exchange ofexperience (internships) & Collective methods of improving
professional

physical skills & Difficult to answer \\
& 18,1 & 6,4 & 51,5 & 6,2 & 19.8 \\
\multirow{2}{=}{7). Is, in your opinion, material incentives the most
effective method of stimulating the professional activities of civil
servants?} & Yes & No & Ratheryesthanno & More likely no than yes &
Difficult to answer \\
& 52,5 & 15,7 & 17,2 & 7,3 & 7,3 \vspace{1.2cm}\\
\multicolumn{6}{l}{Note: compiled by the authors based on sociological research data} \\
\end{longtable}

\begin{multicols}{2}
To the question: "Would you agree to change jobs, subject to a reduction
in pay by 25-30\%, but with more interesting responsibilities?" 18.5\%
answered positively and ``more likely than not'' another 17.9\% of
respondents. The results of the survey indicate that more than a third
of current civil servants are ready to sacrifice material income in
order to change their routine duties to more interesting ones.

This is another confirmation of the theory about the importance of
non-monetary factors motivating the professional growth of civil
servants. However, the presence of 33.1\% of those who are not ready to
sacrifice the amount of wages indicates that an effective motivation
system should consist of both monetary and non-monetary factors
motivating the professional development of modern domestic officials.

For a third of active civil servants, recognition from management is of
the greatest importance (35.4\%), and only after it is promotion
(27.1\%). This situation is caused by the fact that promotion up the
career ladder currently depends more on the loyalty of the management
than on the level of professionalism of the employee.

It should be noted that the majority of civil servants are aware that
the prospect of career growth (32.2\%) is one of the main motives for
their professional growth. At the same time, more than half of the
surveyed respondents admit that the most effective way to increase
professional growth is the exchange of experience (internships) -
51.5\%.

In the process of working in the civil service, in our opinion, the
greatest importance should be the opportunity to acquire knowledge and
skills, which is an internal motive, which, according to personal data,
amounted to only 18.1\%.

According to the survey data, 45.5\% of current civil servants and
21.8\% underestimated the role of staff professionalism had the greatest
impact on the decrease in the motivation and professionalism of civil
servants: recruitment and promotion of personnel through acquaintances,
personal loyalty. These negative factors had a significant impact on the
desire of state bodies to improve their professional level.

It also follows from the results of this survey that more than a third
of civil servants either know very little about it (16.5\%) or have not
encountered it at all in practice (17.0\%). This is another confirmation
of the relevance and timeliness of the ongoing research within the
framework of this research work. Therefore, it is necessary to work out
measures to optimize the system of personnel changes, taking into
account the reduction of the influence of the human factor on the
results.

To the question "Is, in your opinion, material incentives the most
effective method of stimulating the professional activities of civil
servants?" more than half (52.5\%) answered "Yes" and only 17.2\%
answered "No". But we consider it necessary to note that financial
incentives account for 52.5\% of respondents, which indicates the
importance of this factor.

The survey data confirm the fact that gradually non-monetary forms of
motivation are also becoming increasingly important. The survey showed a
high level of internal motivation of domestic civil servants. It follows
that satisfaction from the process itself and the result of the work and
the possibility of the most complete self-realization in this particular
activity are important for them. According to the calculations, this
result states that in the motivation system of a modern civil servant,
it is necessary to pay more attention to non-monetary factors that have
taken leading positions and are the most optimal direction for improving
the existing motivation system.

Thus, based on the conclusions of the above analysis, we can judge the
list of factors of labor motivation of civil servants, the impact of
which the heads of public administration organizations should fully take
into account in their work. The specificity of the system of factors of
labor motivation of civil servants lies in the priority of certain
factors. Without resolving the issues of fulfilling physiological and
security needs, it will not be possible to form a system of labor
motivation for civil servants as a whole. But the factors generated by
the needs of a high order should be taken into account to the same
extent from the very beginning of the organization.

In order to overcome these shortcomings, the development plan of the
Agency of the Republic of Kazakhstan for Civil Service Affairs for
2020-2024 provides for the implementation of a number of measures,
including the necessary amount of work on retraining and advanced
training of civil servants {[}17{]}. In this document, the actual
figures are given for 2018-2019, and the planned period is from
2020-2024. Let\textquotesingle s analyze the Section. 3. ``Strategic
directions, macro indicators, goals and target indicators'', subsection
1. ``Strategic direction. Formation of a strategically innovative civil
service'' The level of compliance with the principle of meritocracy in
2019 was 60.1\%, and in 2024 it should reach 61.6\%.

In subsection 1.1 ``Transformation of the civil service. Target
indicators interconnected with budget programs''.

1. The net turnover of civil servants (leaving the civil service system)
amounted to 6.2\% and 6.0\%, respectively, in 2018-2019, and from 7.0\%
to 6.6\% in the planned period, decreasing every year by 0.1 \%. The
main reason for the high turnover was "command movements", when the key
staff of the apparatus was updated during the change of the first
leaders.

2. The share of civil servants of central government bodies who
completed retraining and advanced training courses, within the funds
allocated from the Republican budget for the reporting period in
2018-2019, amounted to 99.7\% and 100\%, respectively, and from 94.0\%
in the planning period up to 96\% increasing every year by 0.5\%.

3. The share of civil servants satisfied with the quality of education
at the Academy in 2018-2019 amounted to 93.0\% and 94.4\%, respectively,
and in the planned period from 94.5\% to 95.0\%, increasing every year
by 0.5 \% except 2021.

4. The share of implemented recommendations of the Ethics Councils in
2018-2019 amounted to 99.0\% and 99.6\%, respectively, and in the
planned period from 91.1\% to 95.0\%, increasing every year by 1.0\%.

5. For executed orders (representations) to eliminate violations
identified as a result of inspections on compliance with the legislation
on civil service and proposals to cancel decisions of state bodies
adopted in violation of the legislation on civil service and other
normative legal acts of the Republic of Kazakhstan in 2018-2019 amounted
to 96.0\% and 97.0\%, respectively, and in the planned period from
92.5\% to 95.0\%, increasing every year by 0.5 and 1.0\%.

6. The share of advanced training seminars and disciplines of
pre-training courses conducted in the state language at the Academy of
Public Administration and the share of modernized subsystems of IIS
"E-kyzmet", taking into account the transition of subsystems to modules
from among those to be modernized and are shown only in the planning
period, respectively, from 30.0 \% to 34.0\% and from 50.0\% to 90.0\%,
increasing every year.

In subsection 2. ``Strategic direction. Improving the quality of the
provision of public services ''The share of restored rights of service
recipients based on the results of violations identified during
inspections on appeals for the reporting period in 2018-2019 amounted to
83.00\% and 88.4\%, respectively, and in the planned period from 86.0\%
to 90, 1\% increasing every year by 1.0\%. In subsection 2.1
"Development of customer focus in the public service" The share of
public services covered by public monitoring of the quality of public
services for the state social order of the Agency, including online in
the total number in the Register of public services.The share of
instructions and recommendations executed by the central state bodies on
the elimination of violations, causes and conditions contributing to
their completion, identified by the results of inspections, as well as
the results of analyzes carried out in relevant areas on compliance with
the law in terms of the provision of public services and are shown only
in the planning period, respectively, from 60.0\% to 62.0\% and from
82.0\% to 84.0\%, increasing by 1.0\% every year. From the above data,
strategic directions, macro indicators, goals and target indicators,
inclusive of 2024, are visible.

According to the planned indicators, by 2021 it was planned to increase
the share of those who completed retraining and advanced training
courses, of those subject to retraining and advanced training, to
95.0\%, while at the end of 2018 this figure was 70.0\%. In order to
achieve such a high level of this indicator and implement a number of
other measures to improve the quality of services provided, in addition
to the administrative resource, it is necessary to generate internal
staff motivation for professional development.

The main criterion for the quality of public services, in our opinion,
is the satisfaction of their external users, that is, the population. In
order to determine this indicator, constant on-line monitoring is
carried out on the website of the Agency of the Republic of Kazakhstan
for Civil Service Affairs and Anti-Corruption. Consider the results of a
survey conducted at the beginning of 2023 by online monitoring, which
are presented in the table.
\end{multicols}

\begin{longtable}[]{@{}
  >{\raggedright\arraybackslash}p{(\columnwidth - 6\tabcolsep) * \real{0.0683}}
  >{\raggedright\arraybackslash}p{(\columnwidth - 6\tabcolsep) * \real{0.7167}}
  >{\raggedright\arraybackslash}p{(\columnwidth - 6\tabcolsep) * \real{0.1148}}
  >{\raggedright\arraybackslash}p{(\columnwidth - 6\tabcolsep) * \real{0.1002}}@{}}
\caption*{Table 2. Results of a survey of the population on the quality of public
services provided} \\
\toprule\noalign{}
\multirow{2}{*}{\begin{minipage}[b]{\linewidth}\raggedright
N No.

P p/n
\end{minipage}} &
\multirow{2}{*}{\begin{minipage}[b]{\linewidth}\raggedright
Questions, answer options
\end{minipage}} &
\multicolumn{2}{>{\raggedright\arraybackslash}p{(\columnwidth - 6\tabcolsep) * \real{0.2150} + 2\tabcolsep}@{}}{%
\begin{minipage}[b]{\linewidth}\raggedright
Meaning of answer
\end{minipage}} \\
& & \begin{minipage}[b]{\linewidth}\raggedright
people
\end{minipage} & \begin{minipage}[b]{\linewidth}\raggedright
\%
\end{minipage} \\
\midrule\noalign{}
\endhead
\bottomrule\noalign{}
\endlastfoot
1.\textsuperscript{1} & Have you encountered a violation of professional ethics by civil
servants of the Agency of the Republic of Kazakhstan for Civil Service
Affairs (unethical behavior)?

- Yes

- No

- Do not interact with the public sector & 14

218

2 & 6,0

93,2

0,8 \\
\hline
2.\textsuperscript{3}  & To improve the ethical culture of public employees need:

- To bring to disciplinary responsibility

- Encourage ethical civil servants in every possible way

- Own option & 53

160

21 & 22,6

68,4

9,0 \\
\hline
3. & Measures taken in the state body for compliance with ethical
standards:

- Efficient, civil servants are correct and respectful towards citizens
and colleagues

- Ineffective, employees allow facts of incorrect behavior in relation
to citizens and colleagues

- Difficult to answer

- Own option & 186

20

24

4 & 79,5

8,5

10,3

1,7 \\
& Note: compiled by the authors based on sociological research data & & \\
\end{longtable}

\begin{multicols}{2}
The survey involved 234 respondents. To the first question, ``Have you
encountered a violation of professional ethics by civil servants of the
Agency of the Republic of Kazakhstan for Civil Service Affairs
(unethical behavior)? "No" answered 93.2\%, "Yes" answered 6\%, which
indicates the ethical behavior of civil servants and 0.8\% of
respondents, indicates little contact between the population and civil
servants.

On the second question: ``In order to improve the ethical culture of
public22.6 respondents answered that it is necessary to ``Bring to
disciplinary responsibility and - 68.4\% of respondents that it is
necessary to encourage ethical civil servants in every possible way and
only 9\% suggested ``Own option'' 9.0\%.

To the third question: ``Measures taken in the state body tocompliance
with ethical standards" - "Effective,..." answered 79.5
respondents,``Ineffective, ... answered 8.5\% of respondents,
``Difficult to answer'' 10.3\% and offered ``Own option'' - 1.7\% of
respondents.

Thus, in order to detail the problems and disclose the situation, we
analyzed the results of monitoring, which reflects the current
situation.

In the current conditions of the development of the civil service, the
question of the effectiveness of the activities of civil servants is
acute. From the foregoing, it can be seen that the effectiveness of the
civil service depends on the quality and efficiency of the work of civil
servants in the field, respectively, on the motivation in the efficiency
of the work of civil servants in the organization.

Today, the motivation of civil servants is carried out mainly through
strict administrative methods, the main attention is paid to monitoring
the performance of functions, compliance with the activities of civil
servants with established norms and procedures, which provokes a formal
attitude to the performance of official duties or their non-performance
{[}18, pp. 63-67{]}.

Prestige as a motive for entering the civil service are: stability of
position, career opportunities, social guarantees, gaining professional
experience, the managerial nature of work, the possibility of more fully
realizing one\textquotesingle s professional qualities and the desire to
bring more benefits to society and the state. Factors that reduce the
attractiveness of the civil service are low wages, lack of job growth
prospects, a difficult psychological environment, overtime workloads, a
tight schedule, the absence of clear criteria for evaluating performance
and the absence of tangible results of work.

Motivation and stimulation of the work of civil servants is a
problematic topic that requires a systematic approach and improvement.
Today, there is a decrease in the overall level of staff motivation,
which includes:

- the predominance of material needs among officials;

- inefficient system of financial incentives for officials;

- insufficient elaboration of socio-psychological incentive mechanisms.

Considering the foreign experience of motivating civil servants, it can
be noted that the main direction of stimulating the work of officials is
to increase the level of their qualifications, as well as to obtain and
improve special competencies. This is typical for European countries,
the USA, China and Japan {[}19{]}.

Currently, Singapore\textquotesingle s civil service is considered one
of the most efficient in Asia. This efficiency is a consequence of
strict discipline, assertiveness and diligence of officials, low level
of corruption, recruitment of the most capable candidates based on the
principles of meritocracy {[}20{]}. The government of the country enjoys
a good reputation, public support, which it achieves through educational
activities and publicity; the discipline of the people taking tough but
necessary measures.

In Germany, to stimulate talented specialists in the civil service, a
system of ``two directions in a career'' is used: either job growth or
work in the same position with a gradual increase in wages {[}21{]}.It
should also be noted that the motivation for professional
self-realization in Germany occurs already at the stage of training
future specialists. In the training programs for civil servants, the
emphasis is not only on the transfer of certain knowledge, but also on
the formation of a certain way of thinking and behavior of officials of
the corresponding rank, on the prestige of the profession, on
efficiency. In Japan, the system of material incentives consists of two
main blocks: the system of promotion of personnel; systems of natural
and monetary incentives.

It seems that the use of positive foreign practice to motivate the
professional activities of civil servants in the Republic of Kazakhstan
will improve the efficiency of the public administration system.The
severity and relevance of the problems of motivation of public service
employees require further analysis, rethinking of traditional concepts
and the development of modern methods for the formation and
implementation of motivational models in a changing environment of
public life. Therefore, we have identified and searched for methods for
solving problems based on a comparative analysis of foreign and domestic
experience in motivating the professional activities of civil servants.

In Kazakhstan, the remuneration of civil servants was limited to wages.
In our opinion, the introduction of a new model based on a factor-point
scale will increase the responsibility and motivation of civil servants
of the Republic of Kazakhstan.

Thus, in the course of the study, it was revealed that the motivation of
civil servants is due to only partial satisfaction of the needs
associated with the content of the labor itself, social utility, status
needs, and the perception of work as a source of livelihood. In this
regard, the prospect of further research is seen in the development of a
system for stimulating and motivating civil servants in the Republic of
Kazakhstan based on positive foreign experience, as well as the
principles of project management.

The analysis carried out allows us to state the following:

- the motivational mechanism in the public service system includes both
material and non-material components;

- the presence of regulatory and legal aspects for motivating the work
of civil servants necessitates the improvement of the salary structure
of civil servants, depending not only on the length of service in the
civil service, but also on other variable parameters;

- the development of the main elements of the motivational mechanism
involves the use of non-standard managerial, integrated approaches, the
adoption of measures to improve the organizational management model with
the corresponding optimization of the function and number of civil
servants through the introduction of information and communication
technologies in the business processes of state bodies.

The current system for evaluating the performance of civil servants is,
at present, more formal and does not reflect the real results of the
evaluation of the professional performance of civil servants, but
reflects only that part of the evaluation that is based on results and
has a great influence of the human factor. This fact is negative.

It opens up opportunities for distorting factual information under the
influence of human factors (the mood of the manager, honesty and
conscientiousness of the employee, priorities and personal attitudes of
immediate supervisors), and what is even more dangerous, the possibility
of using corruption schemes, which runs counter to the priorities for
the development and modernization of the civil service. in the Republic
of Kazakhstan.

In recent years, the civil service system of Kazakhstan has been
transformed, which has led to a revision of the principles and
approaches to managing the personnel of state bodies in an attempt to
form a professional state. In modern conditions, for solving complex
problems, issues of the quality of personnel management at all levels of
government are of paramount importance.

The achievement of a high level of professionalism by civil servants can
be built on the basis of the motivation of this civil servant to
constantly improve their knowledge and professional competence.
Therefore, the factors of internal motivation of an official for
continuous professional development are at the forefront, as it is able
to ensure its implementation. In our opinion, it is difficult to achieve
such interest by will or order. In this regard, we have set a goal to
identify exactly those factors, the impact of which leads to the
formation of civil servants\textquotesingle{} motivation for
professional development. Knowledge of such employee motivation factors
will allow the leadership of state bodies to build an optimal policy for
managing the professional development of their subordinates.

For the corresponding analytical study of the motivation factors of
civil servants that are of interest to us, general quantitative and
qualitative characteristics of the personnel composition and structure
of government bodies were considered. Based on the conclusions of our
analysis, we can talk about the list of factors of labor motivation of
civil servants, the impact of which the heads of public administration
bodies should fully take into account in the work of civil servants. It
should be noted that the assessment system should not be limited only to
the assessment of the labor activity of a civil servant, but should
cover a wider range of assessed indicators that would form the basis of
not only monetary, but also non-monetary motivation.

The main difficulty lies in the absence of a clear system of criteria
for assessing both a state body and an individual civil servant, which
could be taken as a basis. At the civil service level, the following
performance indicators can and are used as quantitative indicators: the
number of requests or letters processed; the number of processed and
sent responses, the number of requests received, etc. The main
difficulty here is that none of the indicators mentioned above or used
in modern practice can reflect the real contribution of a certain
official to the work of a particular state department, body or
institution.

We hope that the new model of public administration will be based on the
principles of a ``hearing'', efficient, accountable, professional and
pragmatic state. As part of the implementation of the Concept, the
mechanisms for finding and attracting talents for public service at the
country level will be improved. At the same time, the emphasis in the
further improvement of selection procedures will be shifted to assessing
the professional and personal competencies of candidates, including with
the involvement of the public and experts {[}7{]}.

In order to achieve the goal of increasing the motivation of civil
servants through the development of professional competence, it is
necessary to develop a certain system of remuneration and incentives,
both material and intangible, through the professional initiative of
employees in raising their level of professional development.

Therefore, the analytical study is aimed at identifying incentive
factors, the use of which leads to an increase in the efficiency of the
system of motivation for the professional activities of civil servants.
In the process of analysis, the main aspects of quantitative and
qualitative indicators characterizing the composition, structure and
dynamics of the personnel apparatus of public administration bodies were
preliminary touched upon, on the basis of which some important trends
were identified.

As a result of an empirical study of the factors influencing the
motivation of the professional activities of civil servants, there are
both monetary and non-monetary factors. The main non-monetary factors,
according to the assessment of the results of the survey according to
the method of K. Zamfir in the modification of A. Rean "Studying the
motivation of professional activity", include factors of internal
motivation, such as: the possibility of the most complete
self-realization and getting satisfaction from the process itself and
the result of work.However, it should be noted that monetary factors are
of great importance in the system of professional activity motivation.
The main monetary factor is the amount of remuneration, which is
directly related to the current evaluation system aimed at determining
the result. Thus, we come to the conclusion that the motivation system
cannot exist independently of the evaluation system. Therefore, the
question arises of the need to integrate assessment and motivation
systems in order to optimize both systems.

The influence of non-monetary factors, especially in the civil service,
should be significant. The importance of the influence of material
factors is undeniable for maintaining a high professional level of
public service. Building the image of a civil servant is impossible
without ensuring the proper level of remuneration. The work of highly
qualified personnel requires an appropriate assessment.The current
system of wages, with its basic salary levels, remains unattractive,
especially at the level of local government bodies. In addition to the
lower level of wages in the regions, there is also no social package
that attracts workers to the central offices of public service bodies.
Thus, non-material incentive factors remain the main and most accessible
way to increase the level of motivation for the professional activities
of civil servants.

It should be emphasized once again that the development of a system of
motivation for the professional activities of civil servants is based on
an integrated approach that provides for the interaction of the system
with the assessment of the effectiveness / efficiency of their
professional activities, which involves the use of material and
non-material methods to stimulate the professional activities of civil
servants. As a result of the introduction and implementation of
techniques, methods and mechanisms, an effective integrated system for a
comprehensive assessment of the activities of civil servants, taking
into account motivation, is obtained.
\end{multicols}
{\bfseries Conclusions.}

\begin{enumerate}
\item
In Kazakhstan, by now, it has been possible to create a stable and
optimal in terms of the number of state administrative apparatus. There
is an understaffing of civil servants, which, in turn, indicates the
presence of gaps in the work of the personnel service of the state
apparatus.

\item
In the civil service in Kazakhstan, there is a tendency towards a
gradual rejuvenation of the state administration apparatus, which
increases the opportunities for training, retraining and advanced
training, as young employees more effectively perceive new knowledge.
There is a gender balance (59.8\%).

\item
The educational level of civil servants is quite high and tends to
further increase. However, we consider it necessary to note that the
increase in the level of education is accompanied by a decrease in the
quality of services provided by civil servants. Consequently,
professional development, being a multifaceted process, is not always
associated with higher education, but directly proportional to the level
of acquired knowledge and skills.

\item
The Strategic Plan of the Agency of the Republic of Kazakhstan for
Civil Service Affairs for 2020-2024 provides for the implementation of a
number of measures, including carrying out the necessary amount of work
on retraining and advanced training of civil servants, as well as
improving the quality of public services from 83\% in 2018 to 90.1\% in
2024.

\item
Recent reforms in the Republic of Kazakhstan modernized the civil
service, including systems of material incentives and career planning
for civil servants through the introduction of a factor-score scale.

\item
The modernization of the civil service involves the improvement of
the internal management system and the introduction of flexible methods
of human resource management based on achieving the best results of
socio-economic development by improving the quality of public services
and attracting qualified specialists.

\item
To this end, the governments of various countries create and develop
systems of material incentives and career planning for civil servants.
The ultimate goal of the ongoing reforms is to create a professional and
efficient public service.

\item
The final point in optimizing the system of motivation for the
professional activities of civil servants in the Republic of Kazakhstan
is the introduction of elements of project management, which defines a
completely new approach to management for the civil service.

\item
The project approach to the management of the professional activities
of civil servants implies not only the presence of a certain deadline
for the performance of work, but also a clear definition of the
necessary resources, which makes it possible to evaluate the
effectiveness of the work of both an individual employee and the team as
a whole.

\item
The integrated system of motivation and performance assessment that
we have considered allows for continuous monitoring of the motivational
profile of a civil servant at all stages of professional activity from
selection to a position to dismissal. And the practical implementation
and effective use of all the identified factors in combination can
create and maintain a positive motivation for civil servants to improve
their educational and professional level.
\end{enumerate}

\begin{center}
{\bfseries References}
\end{center}

\begin{enumerate}
\item
Khaziev, A. O. Features of the system of motivation of public civil
servants. // Young scientist. - 2019. - No. 47 (285). - S. 234-237.

\item
Message of the President of the Republic of Kazakhstan N. Nazarbayev
to the people of Kazakhstan. October 5, 2018. Increasing the well-being
of Kazakhstanis: increasing incomes and quality of LIFE

\item
Moskaleva N.V. Kuzmenkova V.D. Methods of economic research. -
Smolensk: FGBOU Smolensk State Agricultural Academy, 2016 - 86 p.

\item
Motivation of civil servants in Kazakhstan. 04.04.2018..
09.11.2019.//International Journal of Civil Service Reform and
Practice.: p. 71.

\item
Abilmazhinov T.T. The evolution of systems for assessing and
motivating the professional activities of civil servants // Bulletin of
Modern Research.-2018.- No. 1.1 (16).- P. 179-183.

\item
Evseenko V.V.A. Foreign experience of personnel motivation and the
possibility of its application in domestic practice // Manager. - 2022.
- No. 1 (99). - - S. 87-95

\item
RabigaNurbay. Factor-point scale of civil servants: efficiency and
prospects

\item
National report on the state of public service in the Republic of
Kazakhstan. Nur-Sultan, 2022 \url{https://online.zakon.kz} › Document

\item
National report on the state of the civil service in the Republic of
Kazakhstan for 2019 - GOV.KZ \url{https://www.gov.kz} › qyzmet › documents ›
details

\item
November 11, 2020 National report on the state of the civil service
in the Republic of Kazakhstan. - GOV.KZ \url{https://www.gov.kz} › uploads

\item
Improving the management of state bodies. - under the scientific
supervision of Isenova G.K., Kazakhstan, Nur-Sultan, 2020, 161 p.

\item
The plan of the nation "100 concrete steps to implement five
institutional reforms" / ed. ed. A. M. Rakhimzhanova, B. M. Kaipova. -
Nur-Sultan: Library of the First President of the Republic of Kazakhstan
- Elbasy, 2019. - 196 p.

\item
Strategic Plan of the Agency of the Republic of Kazakhstan for Civil
Service Affairs and Anti-Corruptionfor 2017-2021. Astana, 2017, - 26 p.

\item
Baiturina G.R. Transformation of the Weberian bureaucracy in the
model of R. Merton and A. Gouldner Interpretation of bureaucracy by A.
Gouldner 2018) https://cyberleninka.ru › article ›
transformatsiya-veber.

\item
Maslow A.G. Theory of human motivation / per. in Russian S.A.
language Chetvertakova,2013//
\url{http://www.sergeychet.narod.ru/bibl\_psy/hummotiv1943.htm}. (Date of
access 26.01.2017).

\item
Methodology for studying the motivation of professional activity
(methodology of K. Zamfir in the modification of A.
Rean) \url{http://psihdocs.ru} › metodika-izuchenie-motivacii-profe...
(accessed March 7, 2021)

\item
Appendix to the order of the Chairman of the Agency of the Republic
of Kazakhstan for Civil Service Affairs for 2020-2024 (dated July 21,
2022 No. 163)

\item
Fofanova A. Yu. Criteria for assessing the effectiveness of labor
motivation of civil servants in an organization // CETERIS PARIBUS -
2016 - No. 1-2 - P. 63-67.

\item
Belova N.P., Miroshnichenko O.N. Foreign experience of motivating
the professional activities of civil servants // Bulletin of the Russian
University of Cooperation -2019.- No. 3 (37) - P. 28-31.

\item
Morgan D.G. The Singapore Constitution: A Brief Introduction /
Singapore Management University //
\url{http://www.smu.edu.sg/faculty/profile/9491/Jeremy-}
GOH.

\item
CebroYu.A. Modern problems of motivation and stimulation of labor of
state civil servants // Innovations and investments. - 2019. - No.
3.-S.141-144.
\end{enumerate}

\begin{center}
\emph{{\bfseries Information about authors}}
\end{center}

\begin{itemize}
\item
Karbetova Zatira Rakhimovna - Ph.D., Professorof Economics, Kazakh
University of Technology and Business, Department of Economics and
Service, e-meil kzr\_2011@mail.ru

\item
Baktymbet Asem Serikovna - Candidateof Economics, AssociateProfessor,
Head. Department of "Economics and Management" of theKazakh University
of Technology and Business, e-meil asem\_abs@mail.ru

\item
Karbetova Sholpan Rakhimovna - Ph.D. in Economics, AssociateProfessor,
International Transport and Humanitarian University, Department of
Business and Management, e-meil
\href{mailto:karbetova_2013@mail.ru}{\nolinkurl{karbetova\_2013@mail.ru}}
\end{itemize}

\begin{center}
\emph{{\bfseries Сведения об авторах:}}
\end{center}

\begin{itemize}
\item
Карбетова Затира Рахимовна - кандидат технических наук, профессор,
Казахский университет технологии и бизнеса, факультет экономики и
сервиса, e-meil kzr\_2011@mail.ru

\item
Бактымбет Асем Сериковна - кандидат экономических наук, доцент,
заведующий кафедрой «Экономика и управление» Казахского университета
технологии и бизнеса.

\item
Карбетова Шолпан Рахимовна - кандидат экономических наук, доцент,
Международный транспортно-гуманитарный университет, факультет бизнеса и
менеджмента, e-meil karbetova\_2013@mail.ru
\end{itemize}

\newpage

\begin{center}
Редактор: Оспанова М.К.

Верстка на компьютере:Ундасынов Р. Е.

Подписано в печать 29.06.2023 г.

Издание АО «КазУТБ» 010000, Астана, Казахстан,

ул. Кайыма Мухамедханова, 37 А,

телефон рабочий + (7172) 72-58-12 (134)

е-mail: vestnik@kaztbu.kz
\end{center}
